\documentclass{article}
\usepackage{ctex}
\usepackage{fontspec}
\usepackage{amsmath}
\usepackage{graphicx}

\title{Solution}
\author{Dez}
\date{\today}

\begin{document}
\maketitle

\section{Star Way To Heaven}
题目要求最大化最小距离。考虑二分答案判断是否合法。
把每个星星看作一个圆,我们首先观察:只有一串圆把上下边界连接起来,我们才不能从起点到终点。
而此时答案就是这一串圆中最大的圆的半径(画个图有助于理解)。
这题有一个很神奇的解法,可以把它看作一个最小生成树的题去做。
假设每个星星之间都有边,不必真的把边都连出来,使用$Prim$算法即可。
又因为这里不能用堆去优化,我们每次找最小边都很不方便。
有一种方法是记录每个点与已经在集合中的点的最小距离,每在点集中添加一个点,
就用它的距离试着更新剩下的点,这样每次更新都是$O(n)$的复杂度。
每次选一个点也是$O(n)$,直到上边界和下边界联通为止(把上边界和下边界看作两个点,距离特殊计算,联通用并查集判断)。
最终复杂度$O(n^2)$。

\section{God Knows}
极长上升子序列(占坑)。

\section{Lost My Music}
每个点都带一个权值$c$,题目要求出对于每个点$u$,找出一个祖先$v$最小化$\frac{c_v-c_u}{dis(u,v)}$。
首先给答案式变形,记$d$为节点的深度:
$$\frac{c_v-c_u}{dis(u,v)} = -\frac{c_u-c_v}{d_u-d_v}$$
我们把$(c_i,d_i)$看作一个坐标,那么答案式很像一个斜率的形式。
\newline 然后跳过一段(我不会证)。。。\newline
然后就成为了求凸包切线,用单调栈求解。
但这不是一个序列,而是在树上,所以我们的栈需要可持久化。
又是一个很妙的办法,我们把栈弄成一个类似于链表的形式。
记数组$l[]$为当前下标在栈中下一个元素,再加上一个神奇的倍增跳栈,
每次用倍增找出当前节点$u$前面最优的祖先$v$,用$v$更新$u$的答案,然后令$l[u]\leftarrow v$。
这样的正确性与具体转移需要前面的证明(意思就是我也不懂),
我们就可以在$O(nlog_2n)$的复杂度内求得正解。


\end{document}