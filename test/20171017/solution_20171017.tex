\documentclass{article}
\usepackage{amsmath}
\usepackage{graphicx}
\usepackage{fontspec}
\usepackage{ctex}

\title{Solution}
\author{Dez}
\date{\today}

\begin{document}
\maketitle

\section{斜率}
水题,发现最大斜率一定是由相邻的两个点形成。
可以使用反证法证明。假设最大斜率的两点中间还存在其它的点,
如果它在这两点连线下方,那么它和后面点连线的斜率更优。
如果在上面,则和前面的点连线更优,在线上也不会更差。
因此我们把点按横坐标排序,考虑相邻两点的斜率即可。
时间复杂度$O(n\log n)$。

\section{最优路线}
这个最短路还包含有点权。
我们考虑把点按点权从小到大排序,
按照这样的顺序$Floyd$依次用点去更新当前答案,就能够得到正解。
时间复杂度$O(n^3)$。

\section{小$G$的线段树}
观察题中的两种操作:区间赋值和区间加。
如果最后一个操作是区间赋值的话,那么前面的所有操作都是没有效果的,
而每个赋值操作排最后的概率都是一样的(这里是已经确定最后一个不是加操作),
所以所有赋值操作的期望就是它们的平均数。
再来考虑区间加,区间加操作只有放在所有赋值操作之后才会有贡献,
所以每个加操作起作用的概率等于他排在它加上所有赋值操作的序列的最后一位的概率。
记第$i$个点被赋值了$x_i$次,赋值的总和为$a_i$,加的总和为$b_i$,
则最后答案为$\frac{a_i}{x_i}+\frac{b_i}{x_i+1}$。
注意:如果没有赋值操作,$x_i=0$,但是$a_i$应该等于它原本的权值。
用线段树维护上述三个值,时间复杂度$O(n\log n)$,如果用差分思想维护,可以优化到$O(n)$。

\end{document}