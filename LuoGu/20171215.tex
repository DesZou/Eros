\documentclass{article}
    \usepackage{ctex}
    \usepackage{graphicx}
    \usepackage{fontspec}

\title{Dynamic Programming Review}
\author{Dez}
\date{\today}

\newfontfamily\mono{Source Code Pro}

\begin{document}
\maketitle

\section{消防局的设立}
最后用贪心做了这题。把所有的点按深度放入堆中,每次选出一个深度最大的点,让它的父亲的父亲被选中,这样应该是最优的。
然而实际上实现感觉还是有问题的,数据太水了,没有出错。

\section{小a和uim之大逃离}
设$f[i][j][k][l]$为在位置$(i,j)$小a比uim多吸收$k$点,当前格子由$l$[0:小a,1:uim]吸收的方案数。
把每个格子$(i,j)$的$f[i][j][a[i][j]][0]$初始化为$1$,其中$a[i][j]$表示这个格子的点数。

\section{有线电视网}
树形动归。设$f[i][j]$为以点$i$为根的子树中选取$j$个用户所能获得的最大利益。
那么这就成了一个树形的背包问题,只要记住$j$的枚举要从大到小,还有叶子节点要特判即可。

\section{垃圾陷阱}
非常奇怪。可以枚举的变量太多了,反而不好下手。
设$f[i]$为到高度$i$时所能存活的最大时间。
那么每碰到一个垃圾,都有两种选择,要么吃,要么踩。
并且踩和吃的前提都是你当前状态的存活时间能到达它的下落时间。
所以我们先把垃圾按时间从小到大排序。
然后枚举高度,$f[i]=f[i]+r.t,f[i+r.h]=max(f[i+r.h],f[i])$。
当$i+r.h\geq$要求的高度时,答案即为$r$的下落时间。
如果不能到达指定高度,那么输出$f[0]$,也就是把垃圾全都吃掉。
注意初始化$f[0]=10$。
\end{document}